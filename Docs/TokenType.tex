\subsubsection{Token types}

Token types are important to the interpreter because they differentiate tokens and thus the operations that should be
performed on them and how.


As an example NUMBER tokens can be sumed, because literal expressions build on top of them, but cant be called, which
is a privillege of call expressions, whose main part is an IDENTIFIER token.
So not any token can be sumed, just NUMBER tokens, and not any token can be called, just IDENTIFIER tokens.


Token types are defined on the TokenType.cs file using an enumeration.

\begin{table}
    \caption{Types and its source code representations}
    \begin{tabular}{| l | p{3in} |}
        \hline \\
        TokenType & Represents \\
        \hline
        LEFT\_PAREN & \verb|(|  \\
        RIGHT\_PAREN & \verb|)| \\
        COMMA & \verb|,| \\
        MINUS & \verb|-| \\
        PLUS & \verb|+| \\
        SEMICOLON & \verb|;| \\
        SLASH & \verb|/| \\
        PERCENT & \verb|%| \\
        STAR & \verb|*| \\
        CARET & \verb|^| \\
        AT & \verb|@| \\
        AMPERSAND & \verb|&| \\
        PIPE & \verb!|! \\
        BANG & \verb|!| \\
        BANG\_EQUAL & \verb|!=| \\
        EQUAL & \verb|=| \\
        EQUAL\_EQUAL & \verb|==| \\
        LESS & \verb|<| \\
        LESS\_EQUAL & \verb|<=| \\
        GREATER & \verb|>| \\
        GREATER\_EQUAL & \verb|>=| \\
        ARROW & \verb|=>| \\
        IDENTIFIER & A piece of consecutive letters, digits and underscores where the first character must be a letter or an undescore. \\
        STRING & A string is all the text between two quotes, so the quotes are necesary and the text is optional. \\
        NUMBER & A number is a piece of consecutive digits, optionally followed by a dot '.' and one or more digits. \\
        ELSE & else \\
        FALSE & false \\
        FUNCTION & function \\
        IF & if \\
        TRUE & true \\
        LET & let \\
        IN & in \\
        PI & PI \\
        EULER & E \\
        EOF & Special token type for delimiting the end of the source code. \\
        \hline
    \end{tabular}
\end{table}